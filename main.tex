\documentclass[dvipsnames,table,portrait]{tikzposter}

\usepackage[utf8]{inputenc} % unicode support
\usepackage{lmodern} % enable different font sizes
\usepackage{stmaryrd} % more symbols
\usepackage{amsmath,amsfonts,amssymb} % maths
\usepackage{mathtools} % maths operators
\usepackage{faktor} % quotient rings
\usepackage{graphicx} % graphics
\usepackage{float} % define floating objects like tables or figures
\usepackage{tikz} % draw images
\usepackage{caption} % table caption
\usepackage{booktabs} % table lines
\usepackage{multirow} % table cell merge
\usepackage{multicol} % multicolumns for itemize
\usepackage{aliascnt} % alias counter package
\usepackage{url} % website links
\usepackage{csquotes} % quotes
\usepackage[linesnumbered, ruled]{algorithm2e} % algorithm package
\usepackage[backend=biber,sorting=none,style=numeric]{biblatex} % references
\usepackage{lipsum}

% themes: Default, Rays, Basic, Simple, Envelope, Wave, Board, Autumn, and Desert
\usetheme{Rays}

% counters
\newaliascnt{mycounter}{figure} % let "counter" be an alias for "figure"
\makeatletter
	\let\c@table=\c@mycounter % use mycounter for tables
	\let\c@algocf=\c@mycounter % use mycounter for algocf
\makeatother
\newcommand\addtag{\refstepcounter{mycounter} \tag{\themycounter}} % use mycounter for equations

% maths operators
\DeclareMathOperator{\ord}{ord}
\DeclareMathOperator{\poly}{poly}
\DeclarePairedDelimiter{\abs}{\lvert}{\rvert}
\DeclarePairedDelimiter{\norm}{\lVert}{\rVert}
\DeclarePairedDelimiter{\paren}{(}{)}
\DeclarePairedDelimiter{\bkt}{[}{]}
\DeclarePairedDelimiter{\bbkt}{\llbracket}{\rrbracket}
\DeclarePairedDelimiter{\set}{\{}{\}}
\DeclarePairedDelimiter{\innerprod}{\langle}{\rangle}

% tikz
\usetikzlibrary{shapes.geometric, arrows, chains, calc}
\tikzstyle{shape1} = [ellipse, minimum width=2em, minimum height=1em, text centered, draw=red, fill=gray!10]
\tikzstyle{shape2} = [rectangle, minimum width=3em, minimum height=1cm, text centered, draw=blue, fill=gray!10]
\tikzstyle{arrow} = [thick,->,>=stealth]
\tikzstyle{boxcell}=[draw, minimum size=2em]

% algorithms
\SetKwInOut{Initialization}{Initialization}

% bibliography
\addbibresource{references.bib}

% Title, Author, Institute
\title{Title}
\author{Sim Jun Jie}
\institute{
	School of Physical \& Mathematical Sciences, Nanyang Technological University, Singapore \\
	Institute for Infocomm Research, Agency for Science, Technology and Research (A*STAR), Singapore \\
	\texttt{junjiesim92@gmail.com}
}
% \titlegraphic{LogoGraphic Inserted Here}

\begin{document}

% Title block with title, author, logo, etc.
\maketitle[width = 0.95\textwidth]

\node[anchor=west, yshift=3em] at (TP@title.west) {\includegraphics[width=25em]{ntulogo}};
\node[anchor=east, yshift=3em] at (TP@title.east) {\includegraphics[width=25em]{i2rlogo}};

\begin{columns}
	%%%%%%%%%% First Column %%%%%%%%%%
	\column{0.5}

	\block[titleleft]{Figures and Tables}
	{
		Wrap figures and tables within a tikzfigure environment.

		\begin{tikzfigure}[]
			\captionof{table}{Table Caption}
			\begin{tabular}{lcccc}
				\toprule
				& col $1$ & col $2$ & col $3$ & col  $4$ \\
				\midrule
				row $1$ & $(1,1)$ & $(1,2)$ & $(1,3)$ & $(1,4)$ \\
				row $2$ & $(2,1)$ & $(2,2)$ & $(2,3)$ & $(2,4)$ \\
				row $3$ & $(3,1)$ & $(3,2)$ & $(3,3)$ & $(3,4)$ \\
				\bottomrule
			\end{tabular}
		\end{tikzfigure}
	}

	\block[titleleft]{Inner Blocks}
	{
		\innerblock{This is an inner block}
		{
			Inner blocks can be constructed like that.
		}
		\lipsum[1-1]
		\vspace{0.5em}
		\innerblock{This is another inner block}
		{
			Second inner block.
		}
		\lipsum[2-2]
	}

	%%%%%%%%%% Second Column %%%%%%%%%%
	\column{0.5}

	\block[titleleft]{Sample Block}
	{
		\lipsum[3-5]
	}

	\block[titleleft]{Images}
	{
		\begin{tikzfigure}[]
			\captionof{figure}{Some Image}
			\includegraphics[width=30em]{ntulogo.png}
		\end{tikzfigure}
	}

	\block[titleleft]{References}
	{
		\begin{minipage}{\linewidth}
		\nocite{*}
		\printbibliography[heading=none]
		\end{minipage}
	}
\end{columns}

\end{document}

% \block[titleleft]{Riemann Roch Spaces}
% {
% 	Let $q$ be a prime and $\mathcal{C}$ be an algebraic curve with genus $g$.
% 	Define an algebraic function field $\mathbb{F}_q\paren*{\mathcal{C}}$ as the field extension containing all fractions of polynomials from the polynomial ring $\mathbb{F}_q\bkt*{C}$ (Adding fractions to $\mathbb{Z}$ to get the field $\mathbb{Q}$).

% 	Each algebraic function field is analogous to an algebraic curve.
% 	An equivalence class can be can be induced when considering a special map known as a valuation $\nu: \mathbb{F}_q\paren*{\mathcal{C}} \rightarrow \mathbb{Z}$.
% 	Each equivalence class is called a place.
% 	We can use the Hasse-Weil bound to compute the upper bound of number of places.
% 	There is a one-to-one mapping between points on an algebraic curve and places of the function field.
% 	A divisor is a formal sum of places.
% 	A Riemann Roch space is a special set of polynomials defined by a divisor.
% 	The Riemann-Roch theorem states that for a divisor $G$, $\dim\paren*{G}=\deg\paren*{G}+1-g$.
% }

% \block[titleleft]{HE Parameter Choice}
% {
% 	Powers-of-$2$ cyclotomics are chosen for the plaintext space $R_t$ as ring operations are faster using negacyclic FFT algorithms.
% 	But this results in a large prime $t$.
% 	\begin{tikzfigure}[]
% 		\centering
% 		\begin{tabular}{cccc}
% 			\toprule
% 			$n$ & prime, $t$ & max slots, $\ell$ & degree, $d$ \\
% 			\midrule
% 			\multirow{5}{*}{$4096$}
% 			& $3$ & $2$ & $2048$\\
% 			& $7$ & $4$ & $1024$ \\
% 			& $127$ & $64$ & $64$ \\
% 			& $12799$ & $16$ & $256$ \\
% 			& $40961$ & $4096$ & $1$ \\
% 			\bottomrule
% 		\end{tabular}
% 	\end{tikzfigure}
% }